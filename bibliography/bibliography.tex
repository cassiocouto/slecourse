\documentclass[12pt]{article}

\usepackage[colorlinks=true,urlcolor=blue]{hyperref}

\newcommand{\added}[1]{Entry added by \emph{#1}.}
\newcommand{\available}[1]{Paper available online at \emph{\url{#1}}.}

\begin{document}

\title{\emph{SLECOURSE}: Annotated Bibliography\thanks{\url{http://slecourse.github.com/slecourse}}}

\author{License:  Creative Commons Attribution\\\url{http://freedomdefined.org/Licenses/CC-BY}}

\date{\today}

\maketitle

\begin{abstract}
The purpose of this annotated bibliography is to collect papers that could serve as background for concrete SLE courses. At this early stage of a bibliography, we may be well advised on focusing on more general papers that are likely to be useful for different SLE course designs as opposed to highly technical and specialized work. Should it happen that the bibliography contains confusingly many entries, then we can still impose some extra structure on the bibliography, e.g., based on grouping or tagging.
\end{abstract}

\nocite{*}

\bibliographystyle{plain-annote}
\bibliography{bibliography}
\end{document}