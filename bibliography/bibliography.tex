\documentclass[12pt]{article}

\usepackage[colorlinks=true,urlcolor=blue]{hyperref}

\newcommand{\added}[1]{Entry added by \emph{#1}.}
\newcommand{\public}[1]{Paper publicly available online at \emph{\url{#1}}}
\newcommand{\closed}[1]{Paper non-publicly available online at \emph{\url{#1}}}

\begin{document}

\title{An Annotated SLE Bibliography}

\author{The \emph{SLECOURSE} project\thanks{\url{http://slecourse.github.com/slecourse}}\\
License:  \emph{Creative Commons Attribution}\\\url{http://freedomdefined.org/Licenses/CC-BY}}

\date{Version: \emph{\today}}

\maketitle

\begin{abstract}
  Software Language Engineering (SLE) is a particular view on Software
  Engineering (SE), which pays specific attention to the many software
  languages that are used in software development.  These are not just
  programming languages, but also modeling languages, query and
  transformation languages, schema languages, and domain-specific
  languages. Thus, SLE is concerned with design, implementation,
  testing, deployment, and evolution of software languages as well as
  language-based software components.

  \medskip

  The purpose of this annotated bibliography is to collect papers that
  could serve as background for concrete SLE courses. The bibliography
  may serve well beyond its purpose of supporting teaching. At this early
  stage of a bibliography, we may be well advised on focusing on more
  general papers that are likely to be useful for different SLE course
  designs as opposed to highly technical and specialized work. Should
  it happen that the bibliography contains confusingly many entries,
  then we can still impose some extra structure on the bibliography,
  e.g., based on grouping or tagging.
\end{abstract}

\newpage

\nocite{*}

\bibliographystyle{plain-annote}
\bibliography{bibliography}
\end{document}